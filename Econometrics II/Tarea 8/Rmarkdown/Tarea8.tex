% Options for packages loaded elsewhere
\PassOptionsToPackage{unicode}{hyperref}
\PassOptionsToPackage{hyphens}{url}
%
\documentclass[
]{article}
\usepackage{amsmath,amssymb}
\usepackage{iftex}
\ifPDFTeX
  \usepackage[T1]{fontenc}
  \usepackage[utf8]{inputenc}
  \usepackage{textcomp} % provide euro and other symbols
\else % if luatex or xetex
  \usepackage{unicode-math} % this also loads fontspec
  \defaultfontfeatures{Scale=MatchLowercase}
  \defaultfontfeatures[\rmfamily]{Ligatures=TeX,Scale=1}
\fi
\usepackage{lmodern}
\ifPDFTeX\else
  % xetex/luatex font selection
  \setmainfont[]{Courier}
\fi
% Use upquote if available, for straight quotes in verbatim environments
\IfFileExists{upquote.sty}{\usepackage{upquote}}{}
\IfFileExists{microtype.sty}{% use microtype if available
  \usepackage[]{microtype}
  \UseMicrotypeSet[protrusion]{basicmath} % disable protrusion for tt fonts
}{}
\makeatletter
\@ifundefined{KOMAClassName}{% if non-KOMA class
  \IfFileExists{parskip.sty}{%
    \usepackage{parskip}
  }{% else
    \setlength{\parindent}{0pt}
    \setlength{\parskip}{6pt plus 2pt minus 1pt}}
}{% if KOMA class
  \KOMAoptions{parskip=half}}
\makeatother
\usepackage{xcolor}
\usepackage[margin=1in]{geometry}
\usepackage{color}
\usepackage{fancyvrb}
\newcommand{\VerbBar}{|}
\newcommand{\VERB}{\Verb[commandchars=\\\{\}]}
\DefineVerbatimEnvironment{Highlighting}{Verbatim}{commandchars=\\\{\}}
% Add ',fontsize=\small' for more characters per line
\usepackage{framed}
\definecolor{shadecolor}{RGB}{248,248,248}
\newenvironment{Shaded}{\begin{snugshade}}{\end{snugshade}}
\newcommand{\AlertTok}[1]{\textcolor[rgb]{0.94,0.16,0.16}{#1}}
\newcommand{\AnnotationTok}[1]{\textcolor[rgb]{0.56,0.35,0.01}{\textbf{\textit{#1}}}}
\newcommand{\AttributeTok}[1]{\textcolor[rgb]{0.13,0.29,0.53}{#1}}
\newcommand{\BaseNTok}[1]{\textcolor[rgb]{0.00,0.00,0.81}{#1}}
\newcommand{\BuiltInTok}[1]{#1}
\newcommand{\CharTok}[1]{\textcolor[rgb]{0.31,0.60,0.02}{#1}}
\newcommand{\CommentTok}[1]{\textcolor[rgb]{0.56,0.35,0.01}{\textit{#1}}}
\newcommand{\CommentVarTok}[1]{\textcolor[rgb]{0.56,0.35,0.01}{\textbf{\textit{#1}}}}
\newcommand{\ConstantTok}[1]{\textcolor[rgb]{0.56,0.35,0.01}{#1}}
\newcommand{\ControlFlowTok}[1]{\textcolor[rgb]{0.13,0.29,0.53}{\textbf{#1}}}
\newcommand{\DataTypeTok}[1]{\textcolor[rgb]{0.13,0.29,0.53}{#1}}
\newcommand{\DecValTok}[1]{\textcolor[rgb]{0.00,0.00,0.81}{#1}}
\newcommand{\DocumentationTok}[1]{\textcolor[rgb]{0.56,0.35,0.01}{\textbf{\textit{#1}}}}
\newcommand{\ErrorTok}[1]{\textcolor[rgb]{0.64,0.00,0.00}{\textbf{#1}}}
\newcommand{\ExtensionTok}[1]{#1}
\newcommand{\FloatTok}[1]{\textcolor[rgb]{0.00,0.00,0.81}{#1}}
\newcommand{\FunctionTok}[1]{\textcolor[rgb]{0.13,0.29,0.53}{\textbf{#1}}}
\newcommand{\ImportTok}[1]{#1}
\newcommand{\InformationTok}[1]{\textcolor[rgb]{0.56,0.35,0.01}{\textbf{\textit{#1}}}}
\newcommand{\KeywordTok}[1]{\textcolor[rgb]{0.13,0.29,0.53}{\textbf{#1}}}
\newcommand{\NormalTok}[1]{#1}
\newcommand{\OperatorTok}[1]{\textcolor[rgb]{0.81,0.36,0.00}{\textbf{#1}}}
\newcommand{\OtherTok}[1]{\textcolor[rgb]{0.56,0.35,0.01}{#1}}
\newcommand{\PreprocessorTok}[1]{\textcolor[rgb]{0.56,0.35,0.01}{\textit{#1}}}
\newcommand{\RegionMarkerTok}[1]{#1}
\newcommand{\SpecialCharTok}[1]{\textcolor[rgb]{0.81,0.36,0.00}{\textbf{#1}}}
\newcommand{\SpecialStringTok}[1]{\textcolor[rgb]{0.31,0.60,0.02}{#1}}
\newcommand{\StringTok}[1]{\textcolor[rgb]{0.31,0.60,0.02}{#1}}
\newcommand{\VariableTok}[1]{\textcolor[rgb]{0.00,0.00,0.00}{#1}}
\newcommand{\VerbatimStringTok}[1]{\textcolor[rgb]{0.31,0.60,0.02}{#1}}
\newcommand{\WarningTok}[1]{\textcolor[rgb]{0.56,0.35,0.01}{\textbf{\textit{#1}}}}
\usepackage{graphicx}
\makeatletter
\def\maxwidth{\ifdim\Gin@nat@width>\linewidth\linewidth\else\Gin@nat@width\fi}
\def\maxheight{\ifdim\Gin@nat@height>\textheight\textheight\else\Gin@nat@height\fi}
\makeatother
% Scale images if necessary, so that they will not overflow the page
% margins by default, and it is still possible to overwrite the defaults
% using explicit options in \includegraphics[width, height, ...]{}
\setkeys{Gin}{width=\maxwidth,height=\maxheight,keepaspectratio}
% Set default figure placement to htbp
\makeatletter
\def\fps@figure{htbp}
\makeatother
\setlength{\emergencystretch}{3em} % prevent overfull lines
\providecommand{\tightlist}{%
  \setlength{\itemsep}{0pt}\setlength{\parskip}{0pt}}
\setcounter{secnumdepth}{-\maxdimen} % remove section numbering
\usepackage{amsmath}
\usepackage{setspace}
\usepackage{amssymb}
\usepackage{mathtools}
\ifLuaTeX
  \usepackage{selnolig}  % disable illegal ligatures
\fi
\usepackage{bookmark}
\IfFileExists{xurl.sty}{\usepackage{xurl}}{} % add URL line breaks if available
\urlstyle{same}
\hypersetup{
  hidelinks,
  pdfcreator={LaTeX via pandoc}}

\author{}
\date{\vspace{-2.5em}}

\begin{document}

\subsubsection{i) Generar una muestra
i.i.d.}\label{i-generar-una-muestra-i.i.d.}

\(\begin{Bmatrix}\begin{pmatrix} X_{1i} \\ X_{2i} \\ X_{3i} \end{pmatrix}\end{Bmatrix}_{i=1}^{500}\)
donde
\(\begin{pmatrix} X_{1i} \\ X_{2i} \\ X_{3i} \end{pmatrix} \sim{N_3} \begin{pmatrix} \begin{pmatrix} 1\\ 0\\ 2 \end{pmatrix}, \begin{pmatrix} 0.8 & 0.4 & -0.2\\ 0.4 & 1.0 & -0.8 \\ -0.2 & -0.8 & 2.0 \end{pmatrix}\end{pmatrix}\).

\begin{Shaded}
\begin{Highlighting}[]
\FunctionTok{library}\NormalTok{(mvtnorm)}
\FunctionTok{set.seed}\NormalTok{(}\DecValTok{6}\NormalTok{)}
\FunctionTok{rm}\NormalTok{(}\AttributeTok{list =} \FunctionTok{ls}\NormalTok{())}

\NormalTok{n }\OtherTok{\textless{}{-}} \DecValTok{500}
\NormalTok{mu }\OtherTok{\textless{}{-}} \FunctionTok{c}\NormalTok{(}\DecValTok{1}\NormalTok{, }\DecValTok{0}\NormalTok{, }\DecValTok{2}\NormalTok{)}
\NormalTok{Sigma }\OtherTok{\textless{}{-}} \FunctionTok{matrix}\NormalTok{(}\FunctionTok{c}\NormalTok{(}\FloatTok{0.8}\NormalTok{, }\FloatTok{0.4}\NormalTok{, }\SpecialCharTok{{-}}\FloatTok{0.2}\NormalTok{,}
                  \FloatTok{0.4}\NormalTok{, }\DecValTok{1}\NormalTok{, }\SpecialCharTok{{-}}\FloatTok{0.8}\NormalTok{,}
                  \SpecialCharTok{{-}}\FloatTok{0.2}\NormalTok{, }\SpecialCharTok{{-}}\FloatTok{0.8}\NormalTok{, }\DecValTok{2}\NormalTok{),}
                \AttributeTok{nrow =} \DecValTok{3}\NormalTok{,}
                \AttributeTok{byrow =}\NormalTok{ T)}
\NormalTok{Z }\OtherTok{\textless{}{-}} \FunctionTok{data.frame}\NormalTok{(}
  \FunctionTok{rmvnorm}\NormalTok{(}
\NormalTok{    n,}
    \AttributeTok{mean =}\NormalTok{ mu,}
    \AttributeTok{sigma =}\NormalTok{ Sigma)}
\NormalTok{  )}
\FunctionTok{names}\NormalTok{(Z) }\OtherTok{\textless{}{-}} \FunctionTok{c}\NormalTok{(}\StringTok{"X\_1"}\NormalTok{, }\StringTok{"X\_2"}\NormalTok{, }\StringTok{"X\_3"}\NormalTok{)}

\NormalTok{Z }\OtherTok{\textless{}{-}} \FunctionTok{as.matrix}\NormalTok{(Z)}
\NormalTok{X }\OtherTok{\textless{}{-}} \FunctionTok{cbind}\NormalTok{(}\DecValTok{1}\NormalTok{, Z[,}\SpecialCharTok{{-}}\DecValTok{1}\NormalTok{])}

\NormalTok{n }\OtherTok{\textless{}{-}} \FunctionTok{nrow}\NormalTok{(Z)}
\NormalTok{k }\OtherTok{\textless{}{-}} \FunctionTok{ncol}\NormalTok{(Z)}
\end{Highlighting}
\end{Shaded}

\onehalfspacing

\subsubsection{\texorpdfstring{ii) Generar una muestra
\(\{U_i\}_{i=1}^{500}\) de v.a.s i.i.d. con \(E(U_i) = 0\), tal que
\(U_i\) y el vector aleatorio
\(\begin{pmatrix} X_{1j} & X_{2j} & X_{3j} \end{pmatrix}\) sean
independientes \(\forall i \neq j\), \(i, j \in \{1, 2, \ldots, 500\}\).
Usar cualquier distribución para \(U_i\) que se quiera siempre y cuando
\textbf{no} sea
Normal.}{ii) Generar una muestra \textbackslash\{U\_i\textbackslash\}\_\{i=1\}\^{}\{500\} de v.a.s i.i.d. con E(U\_i) = 0, tal que U\_i y el vector aleatorio \textbackslash begin\{pmatrix\} X\_\{1j\} \& X\_\{2j\} \& X\_\{3j\} \textbackslash end\{pmatrix\} sean independientes \textbackslash forall i \textbackslash neq j, i, j \textbackslash in \textbackslash\{1, 2, \textbackslash ldots, 500\textbackslash\}. Usar cualquier distribución para U\_i que se quiera siempre y cuando  sea Normal.}}\label{ii-generar-una-muestra-u_i_i1500-de-v.a.s-i.i.d.-con-eu_i-0-tal-que-u_i-y-el-vector-aleatorio-beginpmatrix-x_1j-x_2j-x_3j-endpmatrix-sean-independientes-forall-i-neq-j-i-j-in-1-2-ldots-500.-usar-cualquier-distribuciuxf3n-para-u_i-que-se-quiera-siempre-y-cuando-sea-normal.}

\begin{Shaded}
\begin{Highlighting}[]
\NormalTok{U }\OtherTok{\textless{}{-}} \FunctionTok{rlogis}\NormalTok{(n, }\AttributeTok{scale =} \DecValTok{3}\NormalTok{)}
\end{Highlighting}
\end{Shaded}

\subsubsection{\texorpdfstring{iii) Usando los datos obtenidos en i) y
ii), generar la muestra \(\{Y_i\}_{i=1}^500\)
donde}{iii) Usando los datos obtenidos en i) y ii), generar la muestra \textbackslash\{Y\_i\textbackslash\}\_\{i=1\}\^{}500 donde}}\label{iii-usando-los-datos-obtenidos-en-i-y-ii-generar-la-muestra-y_i_i1500-donde}

\[
Y_i = -1 + 2.5X_{1i} - 2X_{2i} + 3X_{3i} + U_i, \quad i = 1, 2, \dots, 500
\]

\begin{Shaded}
\begin{Highlighting}[]
\NormalTok{Y }\OtherTok{\textless{}{-}} \SpecialCharTok{{-}}\DecValTok{1} \SpecialCharTok{+} \FloatTok{2.5}\SpecialCharTok{*}\NormalTok{Z[,}\DecValTok{1}\NormalTok{] }\SpecialCharTok{{-}} \DecValTok{2}\SpecialCharTok{*}\NormalTok{Z[,}\DecValTok{2}\NormalTok{] }\SpecialCharTok{+} \DecValTok{3}\SpecialCharTok{*}\NormalTok{Z[,}\DecValTok{3}\NormalTok{] }\SpecialCharTok{+}\NormalTok{ U}
\end{Highlighting}
\end{Shaded}

\subsubsection{iv) Con los datos obtenidos en i) y iii), estimar por
mínimos cuadrados ordinarios el modelo de regresión
lineal}\label{iv-con-los-datos-obtenidos-en-i-y-iii-estimar-por-muxednimos-cuadrados-ordinarios-el-modelo-de-regresiuxf3n-lineal}

\[
Y_i = b_1 + b_2X_{1i} + b_3X_{2i} + b_4X_{3i} + U_i, \quad i = 1, 2, \dots, 500
\]

\begin{Shaded}
\begin{Highlighting}[]
\NormalTok{regresion }\OtherTok{\textless{}{-}} \FunctionTok{lm}\NormalTok{(Y}\SpecialCharTok{\textasciitilde{}}\NormalTok{X\_1 }\SpecialCharTok{+}\NormalTok{ X\_2 }\SpecialCharTok{+}\NormalTok{ X\_3, }\AttributeTok{data =} \FunctionTok{data.frame}\NormalTok{(Y, Z))}
\FunctionTok{print}\NormalTok{(xtable}\SpecialCharTok{::}\FunctionTok{xtable}\NormalTok{(}\FunctionTok{summary}\NormalTok{(regresion), }\AttributeTok{digits =} \DecValTok{6}\NormalTok{), }\AttributeTok{comment =} \ConstantTok{FALSE}\NormalTok{)}
\end{Highlighting}
\end{Shaded}

\begin{table}[ht]
\centering
\begin{tabular}{rrrrr}
  \hline
 & Estimate & Std. Error & t value & Pr($>$$|$t$|$) \\ 
  \hline
(Intercept) & -0.892335 & 0.525984 & -1.696505 & 0.090418 \\ 
  X\_1 & 2.156798 & 0.309130 & 6.977003 & 0.000000 \\ 
  X\_2 & -1.966675 & 0.316132 & -6.221053 & 0.000000 \\ 
  X\_3 & 3.089357 & 0.203891 & 15.152037 & 0.000000 \\ 
   \hline
\end{tabular}
\end{table}

\subsubsection{\texorpdfstring{v) Encontrar un intervalo de confianza de
\(b_2\) con nivel de confianza de 90\%
(aproximadamente)}{v) Encontrar un intervalo de confianza de b\_2 con nivel de confianza de 90\% (aproximadamente)}}\label{v-encontrar-un-intervalo-de-confianza-de-b_2-con-nivel-de-confianza-de-90-aproximadamente}

\begin{itemize}
  \item[a)] Sin suponer homocedasticidad
\end{itemize}

Podemos utilizar el pivote \[
\dfrac{c'\hat{b_n} - c'b}{\sqrt{c'\frac{1}{n}\hat{V_n}c}} \xrightarrow{\enskip D\enskip} N(0,1)
\] donde \(c=\begin{pmatrix} 0 & 1 & 0 & 0 \end{pmatrix}\) y
\(\hat{V_n} = (\frac{1}{n}X'X)^{-1} (\frac{1}{n}\sum_{j=1}^{n} U_j^2x_jx_j') (\frac{1}{n}X'X)^{-1}\)

Por tanto, con un \(\alpha=10\%\), existe un cuantil \(q\) de la normal
estándar tal que \[
\begin{aligned}
  \mathbb{P}\Bigg[ -q < \dfrac{c'\hat{b_n} - c'b}{\sqrt{c'\frac{1}{n}\hat{V_n}c}} < q \Bigg] = 0.9
  &\iff
  \mathbb{P}\Bigg[ -q\sqrt{c'\frac{1}{n}\hat{V_n}c} < c'\hat{b_n} - c'b< q\sqrt{c'\frac{1}{n}\hat{V_n}c} \Bigg] = 0.9 \\
  &\iff
  \mathbb{P}\Bigg[ - \Big(c'\hat{b_n} + q\sqrt{c'\frac{1}{n}\hat{V_n}c}\Big) <  - c'b < - \Big(c'\hat{b_n} - q\sqrt{c'\frac{1}{n}\hat{V_n}c}\Big) \Bigg] = 0.9 \\
  &\iff
  \mathbb{P}\Bigg[ c'\hat{b_n} + q\sqrt{c'\frac{1}{n}\hat{V_n}c} > c'b > c'\hat{b_n} - q\sqrt{c'\frac{1}{n}\hat{V_n}c} \Bigg] = 0.9 \\
  &\iff
  \mathbb{P}\Bigg[ c'\hat{b_n} - q\sqrt{c'\frac{1}{n}\hat{V_n}c} < c'b < c'\hat{b_n} + q\sqrt{c'\frac{1}{n}\hat{V_n}c} \Bigg] = 0.9 \\
  &\iff
  \mathbb{P}\Bigg[ \hat{b_{n_2}} - q\sqrt{\frac{1}{n}\hat{V_{n_{22}}}} < b_2 < \hat{b_{n_2}} + q\sqrt{\frac{1}{n}\hat{V_{n_{22}}}} \Bigg] = 0.9
\end{aligned}
\] El intervalo de confianza entonces es \[
\Bigg(\hat{b_{n_2}} - q\sqrt{\frac{1}{n}\hat{V_{n_{22}}}} \ \ \ , \ \ \ \hat{b_{n_2}} + q\sqrt{\frac{1}{n}\hat{V_{n_{22}}}}\Bigg)
\] Donde
\(\hat{V_{n_{22}}} = (\frac{1}{n}X'X)^{-1} (\frac{1}{n}\sum_{j=1}^{n} U_j^2x_jx_j') (\frac{1}{n}X'X)^{-1}_{22}\)

Podemos calcularlo:

\begin{Shaded}
\begin{Highlighting}[]
\NormalTok{alpha }\OtherTok{\textless{}{-}} \FloatTok{0.1}
\NormalTok{b\_hat\_2 }\OtherTok{\textless{}{-}}\NormalTok{ regresion}\SpecialCharTok{$}\NormalTok{coefficients[}\DecValTok{2}\NormalTok{]}
\NormalTok{q }\OtherTok{\textless{}{-}} \FunctionTok{qnorm}\NormalTok{(}\AttributeTok{p =}\NormalTok{ alpha}\SpecialCharTok{/}\DecValTok{2}\NormalTok{, }\AttributeTok{lower.tail =}\NormalTok{ F)}
\NormalTok{X }\OtherTok{\textless{}{-}} \FunctionTok{as.matrix}\NormalTok{(}\FunctionTok{cbind}\NormalTok{(}\DecValTok{1}\NormalTok{, Z))}
\NormalTok{XX }\OtherTok{\textless{}{-}} \FunctionTok{t}\NormalTok{(X)}\SpecialCharTok{\%*\%}\NormalTok{X}
\NormalTok{Uj.sq }\OtherTok{\textless{}{-}}\NormalTok{ regresion}\SpecialCharTok{$}\NormalTok{residuals}\SpecialCharTok{\^{}}\DecValTok{2}
\NormalTok{suma\_Uj.sq\_xj\_xj }\OtherTok{\textless{}{-}} \FunctionTok{matrix}\NormalTok{(}\FunctionTok{rep}\NormalTok{(}\DecValTok{0}\NormalTok{, }\DecValTok{4}\NormalTok{), }\AttributeTok{nrow =} \DecValTok{4}\NormalTok{, }\AttributeTok{ncol =} \DecValTok{4}\NormalTok{)}
\ControlFlowTok{for}\NormalTok{ (j }\ControlFlowTok{in} \DecValTok{1}\SpecialCharTok{:}\NormalTok{n) \{}
\NormalTok{  suma\_Uj.sq\_xj\_xj }\OtherTok{\textless{}{-}}\NormalTok{ suma\_Uj.sq\_xj\_xj }\SpecialCharTok{+}\NormalTok{ Uj.sq[j]}\SpecialCharTok{*}\NormalTok{X[j,]}\SpecialCharTok{\%*\%}\FunctionTok{t}\NormalTok{(X[j,])}
\NormalTok{\}}
\NormalTok{mean\_Uj.sq\_xj\_xj }\OtherTok{\textless{}{-}}\NormalTok{ (}\DecValTok{1}\SpecialCharTok{/}\NormalTok{n)}\SpecialCharTok{*}\NormalTok{suma\_Uj.sq\_xj\_xj}
\NormalTok{V\_hat\_hetero }\OtherTok{\textless{}{-}} \FunctionTok{solve}\NormalTok{(}\DecValTok{1}\SpecialCharTok{/}\NormalTok{n}\SpecialCharTok{*}\NormalTok{XX)}\SpecialCharTok{\%*\%}\NormalTok{mean\_Uj.sq\_xj\_xj}\SpecialCharTok{\%*\%}\FunctionTok{solve}\NormalTok{(}\DecValTok{1}\SpecialCharTok{/}\NormalTok{n}\SpecialCharTok{*}\NormalTok{XX)}
\NormalTok{V\_hat\_22\_hetero }\OtherTok{\textless{}{-}}\NormalTok{ V\_hat\_hetero[}\DecValTok{2}\NormalTok{,}\DecValTok{2}\NormalTok{]}
\NormalTok{V\_hat\_hetero}
\end{Highlighting}
\end{Shaded}

\begin{verbatim}
##                      X_1       X_2        X_3
##     145.42992 -35.015880 -24.92654 -43.304769
## X_1 -35.01588  49.912547 -24.80637  -5.198505
## X_2 -24.92654 -24.806367  59.80729  23.078705
## X_3 -43.30477  -5.198505  23.07870  24.356462
\end{verbatim}

\begin{Shaded}
\begin{Highlighting}[]
\NormalTok{CI\_het }\OtherTok{\textless{}{-}} \FunctionTok{c}\NormalTok{(b\_hat\_2 }\SpecialCharTok{{-}}\NormalTok{ q}\SpecialCharTok{*}\FunctionTok{sqrt}\NormalTok{(V\_hat\_22\_hetero}\SpecialCharTok{/}\NormalTok{n),}
\NormalTok{        b\_hat\_2 }\SpecialCharTok{+}\NormalTok{ q}\SpecialCharTok{*}\FunctionTok{sqrt}\NormalTok{(V\_hat\_22\_hetero}\SpecialCharTok{/}\NormalTok{n))}
\NormalTok{CI\_het}
\end{Highlighting}
\end{Shaded}

\begin{verbatim}
##      X_1      X_1 
## 1.637105 2.676491
\end{verbatim}

Es decir, el intervalo de confianza para \(b_2\)
\underline{sin suponer homocedasticidad} es: \[
\Big(1.6371047 , \ \ \ 2.6764914\Big)
\]

\begin{itemize}
  \item[b)] Suponiendo homocedasticidad
\end{itemize}

El intervalo de confianza sigue siendo el mismo, con la diferencia de
que ahora
\(\hat{V_{n_{22}}} = \hat{\sigma}_u^2(\frac{1}{n}X'X)^{-1}_{22}\).
Calculando:

\begin{Shaded}
\begin{Highlighting}[]
\NormalTok{var.residual }\OtherTok{\textless{}{-}} \FunctionTok{summary}\NormalTok{(regresion)}\SpecialCharTok{$}\NormalTok{sigma}\SpecialCharTok{\^{}}\DecValTok{2}
\NormalTok{V\_hat\_homo }\OtherTok{\textless{}{-}}\NormalTok{ var.residual}\SpecialCharTok{*}\FunctionTok{solve}\NormalTok{(}\DecValTok{1}\SpecialCharTok{/}\NormalTok{n}\SpecialCharTok{*}\NormalTok{XX)}
\NormalTok{V\_hat\_22\_homo }\OtherTok{\textless{}{-}}\NormalTok{ V\_hat\_homo[}\DecValTok{2}\NormalTok{,}\DecValTok{2}\NormalTok{]}
\NormalTok{V\_hat\_homo}
\end{Highlighting}
\end{Shaded}

\begin{verbatim}
##                      X_1       X_2        X_3
##     138.32981 -37.069638 -15.15984 -38.127353
## X_1 -37.06964  47.780544 -22.04175  -3.449607
## X_2 -15.15984 -22.041749  49.96977  17.091204
## X_3 -38.12735  -3.449607  17.09120  20.785682
\end{verbatim}

\begin{Shaded}
\begin{Highlighting}[]
\NormalTok{CI\_homo }\OtherTok{\textless{}{-}} \FunctionTok{c}\NormalTok{(b\_hat\_2 }\SpecialCharTok{{-}}\NormalTok{ q}\SpecialCharTok{*}\FunctionTok{sqrt}\NormalTok{(V\_hat\_22\_homo}\SpecialCharTok{/}\NormalTok{n),}
\NormalTok{             b\_hat\_2 }\SpecialCharTok{+}\NormalTok{ q}\SpecialCharTok{*}\FunctionTok{sqrt}\NormalTok{(V\_hat\_22\_homo}\SpecialCharTok{/}\NormalTok{n))}
\NormalTok{CI\_homo}
\end{Highlighting}
\end{Shaded}

\begin{verbatim}
##      X_1      X_1 
## 1.648325 2.665271
\end{verbatim}

Es decir, el intervalo de confianza para \(b_2\)
\underline{suponiendo homocedasticidad} es: \[
\Big(1.6483252 , \ \ \ 2.6652709\Big)
\]

\subsubsection{\texorpdfstring{v) Encontrar un intervalo de confianza de
\(b_1 - 3b_2\) con nivel de confianza de 95\%
(aproximadamente)}{v) Encontrar un intervalo de confianza de b\_1 - 3b\_2 con nivel de confianza de 95\% (aproximadamente)}}\label{v-encontrar-un-intervalo-de-confianza-de-b_1---3b_2-con-nivel-de-confianza-de-95-aproximadamente}

\begin{itemize}
  \item[a)] Sin suponer homocedasticidad
\end{itemize}

En este caso requerimos un pivote que nos permita considerar formas no
lineales de \(b\); podemos utilizar el siguiente: \[
\dfrac{g(\hat{b_n}) - g(b)}{\sqrt{\frac{\partial{g(\hat{b_n})}}{\partial{\hat{b_n}}}\frac{1}{n}\hat{V_n}\frac{\partial{g(\hat{b_n})}}{\partial{\hat{b_n}}}'}} \xrightarrow{\enskip D\enskip} N(0,1)
\]

Donde \(g(\hat{b_n}) = b_1 - 3b_2\),
\(\frac{\partial{g(\hat{b_n})}}{\partial{\hat{b_n}}} = \begin{pmatrix} 1 & -3 & 0 & 0\end{pmatrix}\)
y
\(\hat{V_n} = (\frac{1}{n}X'X)^{-1} (\frac{1}{n}\sum_{j=1}^{n} U_j^2x_jx_j') (\frac{1}{n}X'X)^{-1}\).
Dado esto, el denominador del pivote se convierte en:

\[
\begin{aligned}
  (1,-3,0,0)\frac{1}{n}\hat{V_n}\begin{pmatrix} 1 \\ -3 \\ 0 \\ 0\end{pmatrix}
  &\iff
  \frac{1}{n} \begin{pmatrix} \hat{V}_{n_{11}} - 3\hat{V}_{n_{12}}, & \hat{V}_{n_{12}} - 3\hat{V}_{n_{22}}, & \hat{V}_{n_{13}} - 3\hat{V}_{n_{23}}, & \hat{V}_{n_{14}} - 3\hat{V}_{n_{24}}\end{pmatrix} \begin{pmatrix} 1 \\ -3 \\ 0 \\ 0\end{pmatrix} \\
  &\iff
  \frac{1}{n} \Big(\hat{V}_{n_{11}} - 3\hat{V}_{n_{12}} -3(\hat{V}_{n_{12}} - 3\hat{V}_{n_{22}})\Big) \\
  &\iff
  \frac{1}{n} \Big(\hat{V}_{n_{11}} - 6\hat{V}_{n_{12}} + 9\hat{V}_{n_{22}}\Big)
\end{aligned}
\] El pivote entonces es: \[
\dfrac{g(\hat{b_n}) - g(b)}{\sqrt{\frac{1}{n} \Big(\hat{V}_{n_{11}} - 6\hat{V}_{n_{12}} + 9\hat{V}_{n_{22}}\Big)}} \xrightarrow{\enskip D\enskip} N(0,1)
\]

Por tanto, con un \(\alpha=5\%\), existe un cuantil \(q\) de la normal
estándar tal que \[
\begin{aligned}
  &\mathbb{P}\Bigg[ -q < \dfrac{g(\hat{b_n}) - g(b)}{\sqrt{\frac{1}{n} \Big(\hat{V}_{n_{11}} - 6\hat{V}_{n_{12}} + 9\hat{V}_{n_{22}}\Big)}} < q \Bigg] = 0.95 \\
  &\iff
  \mathbb{P}\Bigg[ -q\sqrt{\frac{1}{n} \Big(\hat{V}_{n_{11}} - 6\hat{V}_{n_{12}} + 9\hat{V}_{n_{22}}\Big)} < g(\hat{b_n}) - g(b) < q\sqrt{\frac{1}{n} \Big(\hat{V}_{n_{11}} - 6\hat{V}_{n_{12}} + 9\hat{V}_{n_{22}}\Big)} \Bigg] = 0.95 \\
  &\iff
  \mathbb{P}\Bigg[ - \Bigg(g(\hat{b_n}) + q\sqrt{\frac{1}{n} \Big(\hat{V}_{n_{11}} - 6\hat{V}_{n_{12}} + 9\hat{V}_{n_{22}}\Big)}\Bigg) <  - g(b) < - \Bigg(g(\hat{b_n}) - q\sqrt{\frac{1}{n} \Big(\hat{V}_{n_{11}} - 6\hat{V}_{n_{12}} + 9\hat{V}_{n_{22}}\Big)}\Bigg) \Bigg] = 0.95 \\
  &\iff
  \mathbb{P}\Bigg[ g(\hat{b_n}) + q\sqrt{\frac{1}{n} \Big(\hat{V}_{n_{11}} - 6\hat{V}_{n_{12}} + 9\hat{V}_{n_{22}}\Big)} > g(b) > g(\hat{b_n}) - q\sqrt{\frac{1}{n} \Big(\hat{V}_{n_{11}} - 6\hat{V}_{n_{12}} + 9\hat{V}_{n_{22}}\Big)} \Bigg] = 0.95 \\
  &\iff
  \mathbb{P}\Bigg[ g(\hat{b_n}) - q\sqrt{\frac{1}{n} \Big(\hat{V}_{n_{11}} - 6\hat{V}_{n_{12}} + 9\hat{V}_{n_{22}}\Big)} < g(b) < g(\hat{b_n}) + q\sqrt{\frac{1}{n} \Big(\hat{V}_{n_{11}} - 6\hat{V}_{n_{12}} + 9\hat{V}_{n_{22}}\Big)} \Bigg] = 0.95
\end{aligned}
\] El intervalo de confianza entonces es \[
\Bigg(g(\hat{b_n}) - q\sqrt{\frac{1}{n} \Big(\hat{V}_{n_{11}} - 6\hat{V}_{n_{12}} + 9\hat{V}_{n_{22}}\Big)} \ \ \ , \ \ \ g(\hat{b_n}) + q\sqrt{\frac{1}{n} \Big(\hat{V}_{n_{11}} - 6\hat{V}_{n_{12}} + 9\hat{V}_{n_{22}}\Big)}\Bigg)
\] Calculando

\begin{Shaded}
\begin{Highlighting}[]
\NormalTok{g\_b\_hat }\OtherTok{\textless{}{-}}\NormalTok{ regresion}\SpecialCharTok{$}\NormalTok{coefficients[}\DecValTok{1}\NormalTok{] }\SpecialCharTok{{-}} \DecValTok{3}\SpecialCharTok{*}\NormalTok{regresion}\SpecialCharTok{$}\NormalTok{coefficients[}\DecValTok{2}\NormalTok{]}
\NormalTok{alpha }\OtherTok{\textless{}{-}} \FloatTok{0.05}
\NormalTok{q }\OtherTok{\textless{}{-}} \FunctionTok{qnorm}\NormalTok{(}\AttributeTok{p =}\NormalTok{ alpha}\SpecialCharTok{/}\DecValTok{2}\NormalTok{, }\AttributeTok{lower.tail =}\NormalTok{ F)}

\NormalTok{CI\_het2 }\OtherTok{\textless{}{-}} \FunctionTok{c}\NormalTok{(g\_b\_hat }\SpecialCharTok{{-}}\NormalTok{ q}\SpecialCharTok{*}\FunctionTok{sqrt}\NormalTok{(}\DecValTok{1}\SpecialCharTok{/}\NormalTok{n }\SpecialCharTok{*}\NormalTok{ (V\_hat\_hetero[}\DecValTok{1}\NormalTok{,}\DecValTok{1}\NormalTok{] }\SpecialCharTok{{-}} \DecValTok{6}\SpecialCharTok{*}\NormalTok{V\_hat\_hetero[}\DecValTok{1}\NormalTok{,}\DecValTok{2}\NormalTok{] }\SpecialCharTok{+} \DecValTok{9}\SpecialCharTok{*}\NormalTok{V\_hat\_hetero[}\DecValTok{2}\NormalTok{,}\DecValTok{2}\NormalTok{])),}
\NormalTok{        g\_b\_hat }\SpecialCharTok{+}\NormalTok{ q}\SpecialCharTok{*}\FunctionTok{sqrt}\NormalTok{(}\DecValTok{1}\SpecialCharTok{/}\NormalTok{n }\SpecialCharTok{*}\NormalTok{ (V\_hat\_hetero[}\DecValTok{1}\NormalTok{,}\DecValTok{1}\NormalTok{] }\SpecialCharTok{{-}} \DecValTok{6}\SpecialCharTok{*}\NormalTok{V\_hat\_hetero[}\DecValTok{1}\NormalTok{,}\DecValTok{2}\NormalTok{] }\SpecialCharTok{+} \DecValTok{9}\SpecialCharTok{*}\NormalTok{V\_hat\_hetero[}\DecValTok{2}\NormalTok{,}\DecValTok{2}\NormalTok{])))}
\NormalTok{CI\_het2}
\end{Highlighting}
\end{Shaded}

\begin{verbatim}
## (Intercept) (Intercept) 
##   -9.849240   -4.876218
\end{verbatim}

Es decir, el intervalo de confianza para \(b_1 - 3b_2\)
\underline{sin suponer homocedasticidad} es: \[
\Big(-9.8492401 , \ \ \ -4.8762182\Big)
\]

\begin{itemize}
  \item[b)] Suponiendo homocedasticidad
\end{itemize}

El intervalo de confianza sigue siendo el mismo, con la diferencia de
que ahora \(\hat{V_n} = \hat{\sigma}_u^2(\frac{1}{n}X'X)^{-1}\).
Calculando:

\begin{Shaded}
\begin{Highlighting}[]
\NormalTok{CI\_homo2 }\OtherTok{\textless{}{-}} \FunctionTok{c}\NormalTok{(g\_b\_hat }\SpecialCharTok{{-}}\NormalTok{ q}\SpecialCharTok{*}\FunctionTok{sqrt}\NormalTok{(}\DecValTok{1}\SpecialCharTok{/}\NormalTok{n }\SpecialCharTok{*}\NormalTok{ (V\_hat\_homo[}\DecValTok{1}\NormalTok{,}\DecValTok{1}\NormalTok{] }\SpecialCharTok{{-}} \DecValTok{6}\SpecialCharTok{*}\NormalTok{V\_hat\_homo[}\DecValTok{1}\NormalTok{,}\DecValTok{2}\NormalTok{] }\SpecialCharTok{+} \DecValTok{9}\SpecialCharTok{*}\NormalTok{V\_hat\_homo[}\DecValTok{2}\NormalTok{,}\DecValTok{2}\NormalTok{])),}
\NormalTok{        g\_b\_hat }\SpecialCharTok{+}\NormalTok{ q}\SpecialCharTok{*}\FunctionTok{sqrt}\NormalTok{(}\DecValTok{1}\SpecialCharTok{/}\NormalTok{n }\SpecialCharTok{*}\NormalTok{ (V\_hat\_homo[}\DecValTok{1}\NormalTok{,}\DecValTok{1}\NormalTok{] }\SpecialCharTok{{-}} \DecValTok{6}\SpecialCharTok{*}\NormalTok{V\_hat\_homo[}\DecValTok{1}\NormalTok{,}\DecValTok{2}\NormalTok{] }\SpecialCharTok{+} \DecValTok{9}\SpecialCharTok{*}\NormalTok{V\_hat\_homo[}\DecValTok{2}\NormalTok{,}\DecValTok{2}\NormalTok{])))}
\NormalTok{CI\_homo2}
\end{Highlighting}
\end{Shaded}

\begin{verbatim}
## (Intercept) (Intercept) 
##   -9.827570   -4.897888
\end{verbatim}

Es decir, el intervalo de confianza para \(b_1 - 3b_2\)
\underline{suponiendo homocedasticidad} es: \[
\Big(-9.82757 , \ \ \ -4.8978883\Big)
\]

\end{document}
